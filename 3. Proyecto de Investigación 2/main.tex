\documentclass[12pt,a4paper]{report}

\usepackage{graphicx}

\usepackage{vmargin}

\usepackage{xcolor}

\usepackage{chicago}

%\setmarginsrb{hleftmargini}{htopmargini}{hrightmargini}{hbottommargini}%

%{hheadheighti}{hheadsepi}{hfootheighti}{hfootskipi}

 

\begin{document}

   

    \begin{titlepage}

               \centering

               %\includegraphics[width=0.15\textwidth]{example-image-1x1}\par\vspace{1cm}

               \includegraphics[width=0.4\textwidth]{udealogo}\par\vspace{1cm}

               {\scshape\LARGE Universidad de Antioquia \par}

               \vspace{1cm}

               {\scshape\large Segundo Proyecto de investigación informática 2\par}

               \vspace{1.5cm}

               {\Large\bfseries Las interrupciones en los microprocesadores: historia y su aplicación en los sitemas actuales\par}

               \vspace{2cm}

               por\par

               {\Large Juan Mauricio Parrado Villa\par}

               %\vfill

               \vspace{2cm}

               Docente\par

               {\Large Augusto Enrique Salazar Jimenez}


               \vfill

    % Bottom of the page

               {\large \today\par}

    \end{titlepage}

    \large

    \begin{center}

        \section*{Las interrupciones en los microprocesadores: historia y su aplicación en los sitemas actuales}

    \end{center}
        
    \vspace{1cm}
       Las interrupciones son una estrategia silenciosa que permite la optimización de los recursos de procesamiento de un microprocesador, respecto a sus dispositivos periféricos.
       
       Éstas se definen como una señal que recibe el CPU para inmediatamente ejecutar un código que está específicamente escrito para responder a la causa de dicha interrupción \cite{Mike Silva 2013}.
       
       Pero vale la vena preguntar, ¿para que querríamos interrumpir un proceso que la CPU se encuentra ejecutando? La respuesta nos indica de donde surgió la necesidad de incluir interrupciones como parte del funcionamiento de un microprocesador. Sólo basta imaginar una CPU que está recibiendo múltiples entradas y salidas desde y hacia los sistemas periféricos, 
       
       
       
    \vspace{1cm}
         
    
        

\begin{thebibliography}{0}
\bibliographystyle{chicago}
\bibliography{bibliografía}

  \bibitem{Mike Silva 2013} Mike Silva, Introduction to Microcontrollers - Interrupts, 2013.
\end{thebibliography}

\end{document}