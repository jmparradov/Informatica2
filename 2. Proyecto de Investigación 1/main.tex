\documentclass[12pt,a4paper]{report}

\usepackage{graphicx}

\usepackage{vmargin}

\usepackage{xcolor}

\usepackage{chicago}

%\setmarginsrb{hleftmargini}{htopmargini}{hrightmargini}{hbottommargini}%

%{hheadheighti}{hheadsepi}{hfootheighti}{hfootskipi}

 

\begin{document}

   

    \begin{titlepage}

               \centering

               %\includegraphics[width=0.15\textwidth]{example-image-1x1}\par\vspace{1cm}

               \includegraphics[width=0.4\textwidth]{udealogo}\par\vspace{1cm}

               {\scshape\LARGE Universidad de Antioquia \par}

               \vspace{1cm}

               {\scshape\large Proyecto de investigación informática 2\par}

               \vspace{1.5cm}

               {\Large\bfseries De la crisis de los fundamentos a la consodilación de la computación moderna\par}

               \vspace{2cm}

               por\par

               {\Large Juan Mauricio Parrado Villa\par}

               %\vfill

               \vspace{2cm}

               Docente\par

               {\Large Augusto Enrique Salazar Jimenez}


               \vfill

    % Bottom of the page

               {\large \today\par}

    \end{titlepage}

    \large

    \begin{center}

        \section*{De la crisis de los fundamentos a la consolidación de la computación moderna}

    \end{center}
        
    \vspace{1cm}
       Se ha evidenciado a través de la hostira humana, que de las grandes crisis que han llevado a la humanidad al límite, casi de su propia autodestrucción, surgen las ideas más brillantes e innovadoras que llevan a nuestra especie hacia un nuevo nivel de conocimiento, impactando directamente nuestra cultura y nuestros hábitos como humanos.Tal y como acontenció en una época particularmente compleja hace ya un siglo, donde se fundaron las bases de la computación moderna, que permitiría el desarrollo de nuestra era de tecnología e informatión, como la conocemos ahora.
       
    \vspace{1cm}
        El inicio del cambio comenzó décadas atrás con el matemático Georg Cantor, quien motivado por su intuición y su pensamiento que algunos llamarían, rebelde o adelantado a su época histórica se atrevió a indegar sobre las profundidades del infinito \cite{Dauben1979}, descubriendo así serias falencias en la fundamentación de la matemática actual, ya que sus resultados daban lugar a paradojas matemáticas, como es el caso de la paradoja de Rusell \cite{Heijenoort1967}.
    \vspace{1cm}
    
        Claramente dichos resultados dió lugar a debates entre los expertos en el tema, discusiones que no parecía tener una solución clara, al contrario, surgían más descubrimientos que desmentían el carácter inapelable de la teoría matemática actual. Así lo hizo Alan Turin, un joven de origen inglés que gracias a su avaricia matemática y su pasión por el aprendizaje, encontró que no es posible definir un sistema lógico que permita discriminar todos y cada uno de los axiomas matemáticos, como falso o verdadero, lo que implica que no existe una certeza real de la veracidad de los mismos \cite{Turing1936}. Un resultado casi insultante para muchos que dedicaban su vida a la verificación de dichos axiomas, como es el caso de David Hilbert, quién pretendía hacer exactamente lo que Turín había confirmado como imposible.
    
    \vspace{1cm}
        Sim embargo los aportes de Turín no terminaron allí, con su notable habilidad para las matemáticas, la computación y la criptografía, Turín en medio de una de las épocas más violentas de nuestra era, definió las bases de la computación moderna, la criptografía e incluso inteligencia artificial \cite{Turing1950}. Durante la segunda guerra mundial, inventó una máquina que se denominó, máquina de Turin, que pondría fin al confilcto, dando una ventaja decisiva a los aliados ya que decifraba los códigos encriptados de la entonces denominada "Enigma", una máquina que consistía en un sistema de engranajes mecánicos, que permitía la encriptación de mensajes, ampliamente usado por el ejército Nazi \cite{Hodges2014}.
    \vspace{1cm}
    
        A partir de entonces y con base en los estudios realizados por Turín, junto con el desarrollo de los sistemas digitales, el modo en que el ser humano se desenvuelve en este medio ha cambiado drásticamente, un cambio que según nuestra histórica, no tiene precedentes y del cual aún todavía sentimos su \textit{momentum}, con el desarrollo de la intelligencia artificial.
        
    \vspace{1cm}
        Partimos así de conceptos gigantes, crisis y guerra, a un recurso computacional que como su orígen, pareciera ser infinito.
        

\begin{thebibliography}{0}
\bibliographystyle{chicago}
\bibliography{bibliografía}

  \bibitem{Dauben1979} Dauben, Joseph W.Georg Cantor: his mathematics and philosophy of the infinite, 1979.
  \bibitem{Heijenoort1967} Heijenoort, J.: From Frege to Göde, 1967; pp. 124-125.
  \bibitem{Turing1936} Turing, A.: On Computable Numbers With an Application to the Entscheidungs, 1936.
  \bibitem{Turing1950}Turing, A.: Computing Machinery and Intelligence, 1950.
  \bibitem{Hodges2014}Hodges, Andrew: Alan Turing: The Enigma, 2014.
\end{thebibliography}

\end{document}