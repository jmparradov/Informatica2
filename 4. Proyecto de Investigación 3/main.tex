\documentclass[12pt,a4paper]{report}

\usepackage{graphicx}

\usepackage{vmargin}

\usepackage{xcolor}

\usepackage{chicago}

%\setmarginsrb{hleftmargini}{htopmargini}{hrightmargini}{hbottommargini}%

%{hheadheighti}{hheadsepi}{hfootheighti}{hfootskipi}

 

\begin{document}

   

    \begin{titlepage}

               \centering

               %\includegraphics[width=0.15\textwidth]{example-image-1x1}\par\vspace{1cm}

               \includegraphics[width=0.4\textwidth]{udealogo}\par\vspace{1cm}

               {\scshape\LARGE Universidad de Antioquia \par}

               \vspace{1cm}

               {\scshape\large Tercer Proyecto de investigación informática 2\par}

               \vspace{1.5cm}

               {\Large\bfseries Los hilos: Su utilidad y evolución en la historia de la computación\par}

               \vspace{2cm}

               por\par

               {\Large Juan Mauricio Parrado Villa\par}

               %\vfill

               \vspace{2cm}

               Docente\par

               {\Large Augusto Enrique Salazar Jimenez}


               \vfill

    % Bottom of the page

               {\large \today\par}

    \end{titlepage}

    \large

    \begin{center}

        \section*{Los hilos: Su utilidad y evolución en la historia de la computación}

    \end{center}
        
    \vspace{1cm}
       Los hilos en el contexto del microprocesador, son la secuencia mínima de comandos que un procesado puede ejecutar.
       
     \vspace{1cm}
     
       Actualmente, la mayoría de los procesadores están construidos para que cada uno de sus núcleos de procesamiento puedan soportar el manejo de varios hilos y así mejorar su eficiencia. Sin embargo la ejecución en hilos múltiples no significa que el procesador maneje varios hilos al mismo tiempo, más se detiene a trabajar en un hilo a la vez, intercambiando entre ellos de acuerdo a las órdenes que recibe del sistema operativo.  \cite{Andrew S. 1992}.
   
   
    

\begin{thebibliography}{0}
\bibliographystyle{chicago}
\bibliography{bibliografía}

  \bibitem{Andrew S. 1992} Andrew S. Tanenbaum, Modern Operating Systems, Second Edition 1992.
  

  
\end{thebibliography}

\end{document}